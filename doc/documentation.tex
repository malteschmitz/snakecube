\documentclass[parskip=half,paper=a4]{scrartcl}

\usepackage[utf8]{inputenc}
\usepackage[ngerman]{babel}
\usepackage[T1]{fontenc}
\usepackage{lmodern}
\usepackage{textcomp}
\usepackage{courier}

% headings without numbers
\setcounter{secnumdepth}{0}

\usepackage{csquotes}
\usepackage{listings}
\lstset{%
  basicstyle=\ttfamily,%
  numbers=none,%
  showstringspaces=false,%
  upquote=true}
\usepackage[toc]{multitoc}
\usepackage{paralist}

\title{Schlangenwürfel}
\subtitle{Programmierprojekt in Artificial Life}
\author{Maximilian Mühlfeld \and Malte Schmitz \and Eike von Tils}

\begin{document}
  \maketitle
  \tableofcontents

  \section{Einleitung}
    Das Ziel des Versuches war es, anhand des Beispiels des Schlangenwürfels einen
    evolutionären (genetischen) Algorithmus zu implementieren. In unserem Fall
    handelt es sich bei den Individuen des Algorithmus um Verdrehungen
    eines Schlangenwürfels in der Ebene. Diese werden durch einen String
    repräsentiert, der für jedes Glied der Schlange die Richtung angibt, in der
    das Folgeglied zu finden ist. Da die Schlange sich im 2D aufhält, gibt es die
    Richtungen \texttt{I} (geradeaus), \texttt{R} (rechts) und
    \texttt{L} (links). Die Energie einer Konfiguration wird aus einer
    Energiefunktion bezogen auf ihre Form und Lage im Raum berechnet.\\
    Hierzu wurde ein Algorithmus nach dem folgenden Schema entwickelt:
    \begin{enumerate}
      \item initiale Population erzeugen 
      \item Bis maximale Epoche oder minimale Energie erreicht wird, wiederhole:
        \begin{enumerate}
          \item Evaluation
          \item Selektion
          \item Reproduktion
        \end{enumerate}
      \end{enumerate}
  
  \subsection[initiale Population]{initiale Population erzeugen}
    Die Initiale Population besteht bei unserem Problem aus einer Menge von
    zufällig erstellten Schlangen. Eine Schlange wird durch einen
    Bitstring repräsentiert. Da die Position der \texttt{I}-Glieder der Schlange
    konstant bleibt, sind nur die Richtungen \texttt{R} und \texttt{L} von
    Bedeutung. Wir codieren diese Richtungen als Richtungsänderungen
    bezüglich der vorherigen Position. Eine 1 steht hierbei für Richtungsänderung,
    eine 0 für gleich bleibende Richtung. Beginn ist immer eine Rechtsdrehung
    \texttt{R}.
    
  \subsection{Evaluation}
    Zur Bewertung der Konfigurationen kommen zwei unterschiedliche
    Energiefunktionen $E_a$ und $E_b$ zum Einsatz.
    \begin{itemize}
    \item $E_a$ berechnet hierbei den Durchmesser des kleinsten Kreises auf der
      Ebene, der die Schlange umschließt.
    \item $E_b$ berechnet die gesamte Fläche aller Löcher, die die Schlange bildet.
    \end{itemize}
    Die genauere Erläuterung zur Implementierung folgt bei der Funktionsbeschreibung.
    
  \subsection{Selektion}
    Die Selektion erfolgt als zufällige Auswahl von Individuen der aktuellen
    Population. Die Wahrscheinlichkeit, das ein Individuum gewählt wird, ist
    hierbei von der Energiefunktion abhängig. Individuen mit niedrigerer Energie
    werden dabei bevorzugt.
    
  \subsection{Reproduktion}
    Reproduktion wird 1. durch Mutation und 2. durch Rekombination ausgeführt.
    
    \begin{enumerate}
      \item Mutation erfolgt durch das zufällige Kippen einer zufälligen Anzahl an
        Bits im Bitstring. Durch die gewählte Kodierung ist sichergestellt, das
        alle folgenden Gelenke mit gedreht werden, sodass die Mutation eines Bits
        genau dem verdrehen eines Gelenks in die andere Richtung entspricht.
      \item Rekombination wird durch das zufälligen Wählen eines Einsprungpunktes
        im Bitstring initiiert. Die beiden, ebenfalls zufällig gewählten,
        Individuen werden an den entsprechenden Stellen gekappt. Ein neues
        Individuum wird dann geboren, bestehend aus einem Teil des ersten
        Individuums bis zum Einsprungpunkt und dem zweiten Teil des
        zweiten Individuums ab dem Einsprungpunkt.
    \end{enumerate}
    
  \section{Funktionsbeschreibung}
    Die Implementierung erfolgte in Ruby. Im folgenden wird auf die einzelnen
    Funktionen und Klassen eingegangen.
 
  \subsection{Klasse \lstinline-Point-}
    Die Klasse \lstinline-Point- ist eine Hilfsklasse, die zur Repräsentation eines
    Punktes im 2D-Raum mit den Koordinaten x und y dient.
    Sie verfügt im Wesentlichen über folgende Methoden:
    \begin{itemize}
      \item \lstinline-+(other)-\\
        Die Methode \lstinline-+- addiert zwei Instanzen von \lstinline-Point-. Hierbei
        wird jeweils auf die Koordinaten \lstinline-x- und \lstinline-y- die der
        übergebenen Instanz addiert, und eine neue Instanz zurückgegeben.
      \item \lstinline-rotate!(d)-\\
        Rotiert die Koordinaten nach links (\texttt{L}) oder rechts (\texttt{R}).
        Die Rotation erfolgt hierbei jeweils um 90$^\circ$.
      \item \lstinline-min(a,b)- bzw. \lstinline-max(a,b)-\\
        Bestimmt die minimale bzw. maximale Lage eines Punktes im Raum bezüglich
        der Koordinatenwerte x und y.
    \end{itemize}
    
  \subsection{Klasse \lstinline-Snake-}
    Die Klasse \lstinline-Snake- dient zur Repräsentation einer Schlange auf einem
    2D-Gitter. Zusätzlich beinhaltet diese Klasse die implementierten
    Energiefunktionen.
    \begin{itemize}
      \item \lstinline-to_board-\\
        Hier wird versucht die Schlange auf ein 2D-Gitter zu legen. Kommt es zu
        Überschneidungen, so wird \lstinline-nil- zurückgegeben. Ansonsten das Board
        mit Schlange. Das zurückgegebene Board ist zudem minimal groß.
      \item \lstinline-to_string-) \\
        Ersetzt die kodierte Darstellung der Schlange (1 für Richtungsänderung,
        0 für gleiche Richtung) durch die Richtungsangaben \texttt{I}, \texttt{R}
        und \texttt{L} anhand eines Templates in \lstinline-snake_static-, indem
        \texttt{R} und \texttt{L} durch \texttt{X} ersetzt sind.
      \item \lstinline-energy_a- \\
        Berechnet die Energiefunktion anhand der Größe des Boards auf dem die
        Schlange liegt. Der Durchmesser des kleinsten Kreises ist identisch mit der
        Diagonalen des Boards. Diese wird mithilfe des Satzes von Pythagoras
        berechnet und zurückgegeben.
      \item \lstinline-energy_b- \\
        Berechnet die Energiefunktion anhand der Anzahl der Löcher. Hierzu wird
        das Board um die Schlange herum mit einem Flood-Filling-Algorithmus gefüllt.
        Alle Stellen auf dem Board, die dann noch leer sind, sind Löcher die von
        der Schlange umschlossen sind. Die Anzahl dieser Stellen wird
        zurückgegeben.
      \item \lstinline-energy_c- \\
        Invertiert die Energiefunktion \lstinline-energy_b-, damit eine
        Minimierung der Energie zu einer maximal großen Lochfläche führt.
    \end{itemize}
  
  \subsection{Klasse \lstinline-Evolution-}
    Die Klasse Evolution implementiert die Funktionalität eines
    evolutionären Algorithmus.
    \begin{itemize}
      \item \lstinline-start-\\
        Die Methode erzeugt eine zufällige Population von Individuen als Binärzahl
        fester Länge mit vorgegebener Anzahl an Individuen.
      \item \lstinline-crossover-\\
        Erzeugt neue Individuen durch Crossover. Hierzu wird ein zufälliger
        Einsprungpunkt gewählt an dem der Crossover stattfindet (s.\,o.).
      \item \lstinline-mutation- \\
        Erzeugt neue Individuen mit zufälligen Mutationen.
        Zuerst wird hierzu eine Bitmaske erstellt, die pro Bitstelle mit einer
        vorgegebenen Wahrscheinlichkeit eine 1 enthält. Diese wird dann mit der
        Kodierten Schlange per \texttt{xor} verrechnet, so dass ein Bit an den
        Stellen gekippt wird, wo die Maske 1 ist.
      \item \lstinline-step-\\
        Führt einen kompletten Iterationsschritt aus. Durch Aufruf von
        \lstinline-crossover- und \lstinline-mutation- werden neue
        Individuen erzeugt.
        
        Anschließend werden diese nach ihrer Energie, gewichtet mit einer
        Zufallszahl, sortiert. Aus dieser sortierten Folge werden die ersten
        Individuen bis zur vorgegebenen Populationsgröße gewählt. Alle
        anderen Individuen sterben aus.
      \item \lstinline-iterate- \\
        Diese Methode startet die komplette Berechnung des Algorithmus.
        Sie ruft die Methode \lstinline-start- auf und dann in jeder
        Iteration einmal \lstinline-step-. Sie iteriert solange, bis entweder die
        maximale Anzahl erreicht wurde, oder eine Energiegrenze unterschritten
        wurde.
      \item \lstinline-bear-\\
        Erzeugt ein neues Individuum bestehend aus dem kodierten Bitstring und
        dem Wert der Energiefunktion.
    \end{itemize}
  
  \subsection[Skript Evolution]{Skript \lstinline-snake_evolution.rb-}
    Dieses Skript erzeugt eine neue Schlange und setzt die Parameter des
    Algorithmus. Es startet anschließend den Algorithmus und gibt die Ergebnisse
    aus. Die Parameter sind dabei im einzelnen:
    \begin{itemize}
      \item \lstinline-length-\\
        Anzahl der Gelenke der Schlange.
      \item \lstinline-energy-\\
        Zu benutzende Energiefunktion.
      \item \lstinline-size-\\
        Größe einer Population.
      \item \lstinline-crossover-\\
        Anzahl an durch Crossover zu erzeugenden
        Individuen pro Iteration.
      \item \lstinline-mutation-\\
        Anzahl an durch Mutationen zu erzeugenden
        Individuen pro Iteration.
      \item \lstinline-flip-\\
        Wahrscheinlichkeit, dass ein Bit bei der Mutation
        gekippt wird.
      \item \lstinline-selection-\\
        Anteil der Zufallskomponente bei der
        Selektion nach Energie der Individuen (Faktor aus dem
        Bereich von 0 bis 1).
      \item \lstinline-n-\\
        Maximale Anzahl an Iterationen.
      \item \lstinline-energy-\\
        Minimale Energie, bei deren Unterschreitung
        die Iteration früher beendet werden kann.
      \item \lstinline-logging-\\
        Flag, das angibt, ob ein Log aller
        Energielevel erzeugt wird.
    \end{itemize}
  
  \subsection[Skript Brute-Force]{Skript \lstinline-snake_bruteforce.rb-}
    Dieses Skript erzeugt die gleiche Schlange, verwendet aber zum Vergleich
    einen Brute-Force-Ansatz, bei dem über alle möglichen Konfigurationen iteriert
    wird. Auf dieses Weise wird das absolute Energie-Minimum sicher gefunden, sodass
    die Ergebnisse des evolutionären Algorithmus besser beurteilt werden können.
  
  \section{Durchführung}
    Das Programm wird auf der Kommandozeile durch den Befehl
    \begin{lstlisting}[gobble=6]
      ruby snake_evolution.rb
    \end{lstlisting}
    gestartet.
    
    Die Ausgabe besteht aus der Anzahl an Iterationen, der endgültigen
    Population sowie einer Liste aller Energielevel, die erreicht wurden und
    jeweils des ersten Individuums mit dieser Energie.
    Am Ende wird die Schlange auf ihrem 2D Gatter gezeichnet.
  
    Als Schlange wird \texttt{IIXXXIXXIXXXIXIXXXXIXIXIXII} verwendet.
    
  \subsection{Energiefunktion $E_a$} 
    Es werden die folgenden Parameter verwendet:
    \begin{compactitem}[--]
      \item \lstinline-length = 27-
      \item \lstinline-energy = energy_a-
      \item \lstinline-size = 15-
      \item \lstinline-crossover = 5-
      \item \lstinline-mutation = 15-
      \item \lstinline-flip = 0.3-
      \item \lstinline-selection =  0.5-
      \item \lstinline-n = 10000-
      \item \lstinline-energy = 9-
      \item \lstinline-logging = true-
    \end{compactitem}
  
    Wir erhalten folgendes Ergebnis:
    \begin{lstlisting}[gobble=6,basicstyle=\small\ttfamily]
      number of iterations:
      52
      
      final population:
      8.602325	1110110110010010	IILRLILRILLRILILLRRIRILILII
      9.899495	0111100101011010	IIRLRILRIRRLILIRRLRIRILILII
      9.899495	0111100101011010	IIRLRILRIRRLILIRRLRIRILILII
      9.899495	0111100101011010	IIRLRILRIRRLILIRRLRIRILILII
      9.899495	1111100101011010	IILRLIRLILLRIRILLRLILIRIRII
      9.899495	0111100101011010	IIRLRILRIRRLILIRRLRIRILILII
      9.899495	0011100101011010	IIRRLIRLILLRIRILLRLILIRIRII
      9.899495	0111100101011010	IIRLRILRIRRLILIRRLRIRILILII
      9.899495	0111100101011010	IIRLRILRIRRLILIRRLRIRILILII
      9.899495	1111100101011010	IILRLIRLILLRIRILLRLILIRIRII
      9.899495	1111100101011010	IILRLIRLILLRIRILLRLILIRIRII
      9.899495	0111100101011010	IIRLRILRIRRLILIRRLRIRILILII
      9.899495	0111100101011010	IIRLRILRIRRLILIRRLRIRILILII
      10.000000	1011100111011011	IILLRILRIRRLIRILLRLILIRILII
      14.212670	1111010101111010	IILRLIRRILLRIRILRLRIRILILII
      14.866069	1011101111110101	IILLRILRIRLRILIRLRRILILIRII
      
      all energies and first found individual with that energy:
      8.602325	1110110110010010	IILRLILRILLRILILLRRIRILILII
      9.899495	0111100101011010	IIRLRILRIRRLILIRRLRIRILILII
      10.000000	0111101101101010	IIRLRILRIRLRIRILRRLILIRIRII
      ...
      17.804494	0110101111101011	IIRLRIRLILRLIRILRRLILIRILII
      Inf	1011000010010101	IILLRILLILLLIRIRRLLIRIRILII
      
      fittest individual in final population:
      8.602325	1110110110010010	IILRLILRILLRILILLRRIRILILII
      
        LL 
      LIRI 
      IRRRL
      LLI I
      LIRRL
      IIIL 
      LIIX 
    \end{lstlisting}
    
    Man erkennt deutlich, dass hier bereits nach wenigen Iterationen
    eine sehr kompakte Konfiguration der Schlange erreicht wurde.
    Der Vergleich mit den Ergebnissen des Brute-Force-Skriptes zeigt,
    dass mit einer Energie von etwa 8.602 ein globales Optimum gefunden
    wird. 
  
  \subsection{Energiefunktion $E_b$} 
    Es werden die folgenden Parameter verwendet:
    \begin{compactitem}[--]
      \item \lstinline-length = 27-
      \item \lstinline-energy = energy_c-
      \item \lstinline-size = 15-
      \item \lstinline-crossover = 5-
      \item \lstinline-mutation = 15-
      \item \lstinline-flip = 0.3-
      \item \lstinline-selection = 0.5-
      \item \lstinline-n = 10000-
      \item \lstinline|energy = -15|
      \item \lstinline-logging = true-
    \end{compactitem}
    
    Wir erhalten folgendes Ergebnis:
    \begin{lstlisting}[gobble=6,basicstyle=\small\ttfamily]
      number of iterations:
      71
      
      final population:
      -15.000000	0010111101111011	IIRRLILRILRLILIRLRLILIRILII
      -13.000000	0110111101111011	IIRLRIRLIRLRIRILRLRIRILIRII
      -13.000000	0110111101111011	IIRLRIRLIRLRIRILRLRIRILIRII
      -13.000000	0110111101111011	IIRLRIRLIRLRIRILRLRIRILIRII
      -13.000000	0110111101111011	IIRLRIRLIRLRIRILRLRIRILIRII
      -13.000000	0110111101111011	IIRLRIRLIRLRIRILRLRIRILIRII
      -13.000000	0110111101111011	IIRLRIRLIRLRIRILRLRIRILIRII
      -13.000000	0110111101111011	IIRLRIRLIRLRIRILRLRIRILIRII
      -13.000000	0110111101111011	IIRLRIRLIRLRIRILRLRIRILIRII
      -10.000000	0110111101010010	IIRLRIRLIRLRIRILLRRIRILILII
      -10.000000	0110111101010010	IIRLRIRLIRLRIRILLRRIRILILII
      -10.000000	0110111101010010	IIRLRIRLIRLRIRILLRRIRILILII
      -10.000000	0110111101010010	IIRLRIRLIRLRIRILLRRIRILILII
      -10.000000	0110111101010010	IIRLRIRLIRLRIRILLRRIRILILII
      -10.000000	0110111101010010	IIRLRIRLIRLRIRILLRRIRILILII
      -10.000000	0110111101010010	IIRLRIRLIRLRIRILLRRIRILILII
      
      all energies and first found individual with that energy:
      -15.000000	0010111101111011	IIRRLILRILRLILIRLRLILIRILII
      -13.000000	0110111101111011	IIRLRIRLIRLRIRILRLRIRILIRII
      ...
      -1.000000	1101110100111110	IILRRILRILLRIRIRLRLIRILILII
      0.000000	1011101011010011	IILLRILRIRLLIRILLRRIRILIRII
      
      fittest individual in final population:
      -15.000000	0010111101111011	IIRRLILRILRLILIRLRLILIRILII
      
        XIIL  
      IIR  I  
       LR  RIL
       I     I
       LR   RL
        I  RL 
        LR I  
         LIL  
    \end{lstlisting}
    
    An der gezeichneten Schlange erkennt man deutlich, dass eine Schleife mit
    großem Loch gefunden wurde. Bei dieser Energiefunktion ist das Ergebnis
    aufgrund der zerklüfteten Energielandschaft (gut erkennbar an den Sprüngen
    in den Energien der Konfigurationen in der finalen Population) allerdings
    stärker von den zufälligen Startwerten abhängig. Entsprechend kann es sehr
    lange dauern -- in ganz schlechten Läufen auch unendlich lange -- bis
    das globale Optimum von $-15$ erreicht wird. 
\end{document}